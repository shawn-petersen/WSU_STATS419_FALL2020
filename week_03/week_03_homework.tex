\documentclass{article}


\topmargin=-0.45in      %
\evensidemargin=0in     %
\oddsidemargin=0in      %
\textwidth=6.5in        %
\textheight=9.0in       %
\headsep=0.25in         %


\usepackage{graphicx,float,wrapfig}
\usepackage{moreverb} % monospace
\usepackage{fancyhdr} % fancy header 
\usepackage{lastpage} % last page 
\usepackage{setspace} % line spacing
\usepackage{amsmath,amsfonts,amsthm,amssymb}  % math fonts 
\usepackage{tabularx} % advanced tables

%%% default RStudio packages %%%
\usepackage[unicode=true]{hyperref}




\usepackage[usenames,dvipsnames]{color}
\usepackage{listings}
% xcolor ... #rrggbb ... 





\usepackage{color}
\usepackage{fancyvrb}
\newcommand{\VerbBar}{|}
\newcommand{\VERB}{\Verb[commandchars=\\\{\}]}
\DefineVerbatimEnvironment{Highlighting}{Verbatim}{commandchars=\\\{\}}
% Add ',fontsize=\small' for more characters per line
\usepackage{framed}
\definecolor{shadecolor}{RGB}{248,248,248}
\newenvironment{Shaded}{\begin{snugshade}}{\end{snugshade}}
\newcommand{\AlertTok}[1]{\textcolor[rgb]{0.94,0.16,0.16}{#1}}
\newcommand{\AnnotationTok}[1]{\textcolor[rgb]{0.56,0.35,0.01}{\textbf{\textit{#1}}}}
\newcommand{\AttributeTok}[1]{\textcolor[rgb]{0.77,0.63,0.00}{#1}}
\newcommand{\BaseNTok}[1]{\textcolor[rgb]{0.00,0.00,0.81}{#1}}
\newcommand{\BuiltInTok}[1]{#1}
\newcommand{\CharTok}[1]{\textcolor[rgb]{0.31,0.60,0.02}{#1}}
\newcommand{\CommentTok}[1]{\textcolor[rgb]{0.56,0.35,0.01}{\textit{#1}}}
\newcommand{\CommentVarTok}[1]{\textcolor[rgb]{0.56,0.35,0.01}{\textbf{\textit{#1}}}}
\newcommand{\ConstantTok}[1]{\textcolor[rgb]{0.00,0.00,0.00}{#1}}
\newcommand{\ControlFlowTok}[1]{\textcolor[rgb]{0.13,0.29,0.53}{\textbf{#1}}}
\newcommand{\DataTypeTok}[1]{\textcolor[rgb]{0.13,0.29,0.53}{#1}}
\newcommand{\DecValTok}[1]{\textcolor[rgb]{0.00,0.00,0.81}{#1}}
\newcommand{\DocumentationTok}[1]{\textcolor[rgb]{0.56,0.35,0.01}{\textbf{\textit{#1}}}}
\newcommand{\ErrorTok}[1]{\textcolor[rgb]{0.64,0.00,0.00}{\textbf{#1}}}
\newcommand{\ExtensionTok}[1]{#1}
\newcommand{\FloatTok}[1]{\textcolor[rgb]{0.00,0.00,0.81}{#1}}
\newcommand{\FunctionTok}[1]{\textcolor[rgb]{0.00,0.00,0.00}{#1}}
\newcommand{\ImportTok}[1]{#1}
\newcommand{\InformationTok}[1]{\textcolor[rgb]{0.56,0.35,0.01}{\textbf{\textit{#1}}}}
\newcommand{\KeywordTok}[1]{\textcolor[rgb]{0.13,0.29,0.53}{\textbf{#1}}}
\newcommand{\NormalTok}[1]{#1}
\newcommand{\OperatorTok}[1]{\textcolor[rgb]{0.81,0.36,0.00}{\textbf{#1}}}
\newcommand{\OtherTok}[1]{\textcolor[rgb]{0.56,0.35,0.01}{#1}}
\newcommand{\PreprocessorTok}[1]{\textcolor[rgb]{0.56,0.35,0.01}{\textit{#1}}}
\newcommand{\RegionMarkerTok}[1]{#1}
\newcommand{\SpecialCharTok}[1]{\textcolor[rgb]{0.00,0.00,0.00}{#1}}
\newcommand{\SpecialStringTok}[1]{\textcolor[rgb]{0.31,0.60,0.02}{#1}}
\newcommand{\StringTok}[1]{\textcolor[rgb]{0.31,0.60,0.02}{#1}}
\newcommand{\VariableTok}[1]{\textcolor[rgb]{0.00,0.00,0.00}{#1}}
\newcommand{\VerbatimStringTok}[1]{\textcolor[rgb]{0.31,0.60,0.02}{#1}}
\newcommand{\WarningTok}[1]{\textcolor[rgb]{0.56,0.35,0.01}{\textbf{\textit{#1}}}}
%%%%%%%%%%%%%%%%%%%%%%%%%%%%%%%%
% https://github.com/svmiller/svm-r-markdown-templates/blob/master/svm-latex-memo.tex


%\usepackage[sort]{natbib}  %% will alpha/numeric in order inline
%	\setcitestyle{aysep={}} %% no year, comma just year
\usepackage[numbers]{natbib}
	
	
\newcommand{\hmwkCourse}{ \textbf{STATS 419 Survey of Multivariate Analysis} }
\newcommand{\hmwkCourseShort}{}
\newcommand{\hmwkTitle}{week 03 Assignment}
\newcommand{\hmwkAuthor}{Shawn Petersen}
\newcommand{\hmwkEmail}{\href{mailto:shawn.petersen@wsu.edu}{\nolinkurl{shawn.petersen@wsu.edu}}}
\newcommand{\hmwkWSU}{[]}
\newcommand{\hmwkInstructor}{Instructor: Monte J. Shaffer}
\newcommand{\hmwkDate}{19 September 2020}


\pagestyle{fancy}   
\lhead{\hmwkCourseShort}
\chead{\hmwkTitle}
\rhead{\hmwkAuthor}
\lfoot{}
\cfoot{Page\ \thepage\ of\ \protect\pageref{LastPage}}
\rfoot{}



\title{\hmwkCourse \\ \hmwkTitle}
\author{\hmwkAuthor \\ (\hmwkEmail) \\ \hmwkWSU \\[0.5in] \hmwkInstructor }
\date{\hmwkDate}


\usepackage{titling}
\pretitle{\begin{flushright}\LARGE}
\posttitle{\par\end{flushright}\vskip 0.5em}
\preauthor{\begin{flushright}\large \lineskip 0.5em}
\postauthor{\par\end{flushright}}
\predate{\begin{flushright}\large}
\postdate{\par\end{flushright}}


%\tracingall




\hypersetup{breaklinks=true,%
            bookmarks=true,%
            pdfauthor={Shawn Petersen},%
            pdftitle={week 03 Assignment},%
            colorlinks=true,%
            citecolor=blue,%
            urlcolor=blue,%
            linkcolor=magenta,%
            pdfborder={0 0 0}%
            }%
            


\begin{document}

\maketitle



\begin{Shaded}
\begin{Highlighting}[]
\KeywordTok{getwd}\NormalTok{()}

\KeywordTok{library}\NormalTok{(devtools)  }\CommentTok{# devtools is requireed for source_url to work}
\end{Highlighting}
\end{Shaded}

\begin{verbatim}
## Loading required package: usethis
\end{verbatim}

\begin{Shaded}
\begin{Highlighting}[]
\NormalTok{my.source =}\StringTok{ 'github'}
\NormalTok{my.github.path =}\StringTok{ "https://raw.githubusercontent.com/shawn-petersen/WSU_STATS419_FALL2020/"}
\KeywordTok{source}\NormalTok{( }\KeywordTok{paste0}\NormalTok{(my.github.path,}\StringTok{"master/libraries/libraries.R"}\NormalTok{) );}
\end{Highlighting}
\end{Shaded}

\begin{verbatim}
## -- Attaching packages ------------------------------------------------------------------------------------------------------------------------------------------------------------- tidyverse 1.3.0 --
\end{verbatim}

\begin{verbatim}
## v ggplot2 3.3.2     v purrr   0.3.4
## v tibble  3.0.3     v dplyr   1.0.2
## v tidyr   1.1.2     v stringr 1.4.0
## v readr   1.3.1     v forcats 0.5.0
\end{verbatim}

\begin{verbatim}
## Warning: package 'ggplot2' was built under R version 3.6.3
\end{verbatim}

\begin{verbatim}
## Warning: package 'tibble' was built under R version 3.6.3
\end{verbatim}

\begin{verbatim}
## Warning: package 'tidyr' was built under R version 3.6.3
\end{verbatim}

\begin{verbatim}
## Warning: package 'purrr' was built under R version 3.6.3
\end{verbatim}

\begin{verbatim}
## Warning: package 'dplyr' was built under R version 3.6.3
\end{verbatim}

\begin{verbatim}
## Warning: package 'forcats' was built under R version 3.6.3
\end{verbatim}

\begin{verbatim}
## -- Conflicts ---------------------------------------------------------------------------------------------------------------------------------------------------------------- tidyverse_conflicts() --
## x dplyr::filter() masks stats::filter()
## x dplyr::lag()    masks stats::lag()
\end{verbatim}

\begin{verbatim}
## Warning: package 'tidyselect' was built under R version 3.6.3
\end{verbatim}

\begin{verbatim}
## Warning: package 'rmarkdown' was built under R version 3.6.3
\end{verbatim}

\begin{verbatim}
## Warning: package 'lubridate' was built under R version 3.6.3
\end{verbatim}

\begin{verbatim}
## 
## Attaching package: 'lubridate'
\end{verbatim}

\begin{verbatim}
## The following objects are masked from 'package:base':
## 
##     date, intersect, setdiff, union
\end{verbatim}

\begin{verbatim}
## Warning: package 'rvest' was built under R version 3.6.3
\end{verbatim}

\begin{verbatim}
## Loading required package: xml2
\end{verbatim}

\begin{verbatim}
## Warning: package 'xml2' was built under R version 3.6.3
\end{verbatim}

\begin{verbatim}
## 
## Attaching package: 'rvest'
\end{verbatim}

\begin{verbatim}
## The following object is masked from 'package:purrr':
## 
##     pluck
\end{verbatim}

\begin{verbatim}
## The following object is masked from 'package:readr':
## 
##     guess_encoding
\end{verbatim}

\begin{verbatim}
## Warning: package 'digest' was built under R version 3.6.3
\end{verbatim}

\begin{Shaded}
\begin{Highlighting}[]
\KeywordTok{source}\NormalTok{( }\KeywordTok{paste0}\NormalTok{(my.github.path,}\StringTok{"master/functions/functions-imdb.R"}\NormalTok{) );}
\KeywordTok{source}\NormalTok{( }\KeywordTok{paste0}\NormalTok{(my.github.path,}\StringTok{"master/functions/myFunctions.R"}\NormalTok{) );}

\CommentTok{#local.path = "D:/_git_/WSU_STATS419_FALL2020/";}
\CommentTok{#source( paste0(local.path,"functions/myFunctions.R"), local=T );}
\end{Highlighting}
\end{Shaded}

\hypertarget{introduction}{%
\section{Introduction}\label{introduction}}

For homework week \#2 we are applying our ``R'' knowledge for matrix
math, data wrangling personality data, creating functions for basic data
analysis, looking at IMDB data from Will Smith \& Denzel Washington and
box plots the results of our movie analysis.

\newpage

\hypertarget{question-1}{%
\section{\texorpdfstring{\textbf{Question
1}}{Question 1}}\label{question-1}}

\hypertarget{rotation-matrix}{%
\subsection{Rotation Matrix}\label{rotation-matrix}}

Create the ``rotate matrix'' functions as described in lectures. Apply
to the example ``myMatrix''

Answer: I created one function and the user is able to pass in the
original matrix and a rotational value in degrees. I support 90, 180,
and 270 degreee rotations with this function

\begin{Shaded}
\begin{Highlighting}[]
\CommentTok{# Function to rotate a matrix based on user passed parameter on degrees}
\NormalTok{transposeMatrix <-}\StringTok{ }\ControlFlowTok{function}\NormalTok{(myMatrix,rotate_degrees)}
\NormalTok{\{}
\NormalTok{   transformation_mat <-}\StringTok{ }\KeywordTok{matrix}\NormalTok{ (}\KeywordTok{c}\NormalTok{ ( }\DecValTok{0}\NormalTok{,  }\DecValTok{0}\NormalTok{,  }\DecValTok{1}\NormalTok{,}
                                     \DecValTok{0}\NormalTok{,  }\DecValTok{1}\NormalTok{,  }\DecValTok{0}\NormalTok{,}
                                     \DecValTok{1}\NormalTok{,  }\DecValTok{0}\NormalTok{,  }\DecValTok{0}\NormalTok{), }\DataTypeTok{nrow=}\DecValTok{3}\NormalTok{, }\DataTypeTok{byrow=}\NormalTok{T);}
  
  \ControlFlowTok{if}\NormalTok{ (rotate_degrees }\OperatorTok{==}\StringTok{ }\DecValTok{90}\NormalTok{)\{}
\NormalTok{    matrixrotate <-}\StringTok{ }\KeywordTok{t}\NormalTok{(myMatrix }\OperatorTok\StringTok{ }\NormalTok{transformation_mat)}
\NormalTok{  \}}\ControlFlowTok{else} \ControlFlowTok{if}\NormalTok{ (rotate_degrees }\OperatorTok{==}\StringTok{ }\DecValTok{180}\NormalTok{)\{}
\NormalTok{    temprotate <-}\StringTok{ }\KeywordTok{t}\NormalTok{(myMatrix }\OperatorTok\StringTok{ }\NormalTok{transformation_mat)}
\NormalTok{    matrixrotate <-}\StringTok{ }\KeywordTok{t}\NormalTok{(temprotate }\OperatorTok\StringTok{ }\NormalTok{transformation_mat)}
\NormalTok{  \} }\ControlFlowTok{else} \ControlFlowTok{if}\NormalTok{ (rotate_degrees }\OperatorTok{==}\StringTok{ }\DecValTok{270}\NormalTok{)\{}
\NormalTok{    temprotate <-}\StringTok{ }\KeywordTok{t}\NormalTok{(myMatrix }\OperatorTok\StringTok{ }\NormalTok{transformation_mat)}
\NormalTok{    temprotate <-}\StringTok{ }\KeywordTok{t}\NormalTok{(temprotate }\OperatorTok\StringTok{ }\NormalTok{transformation_mat)}
\NormalTok{    matrixrotate <-}\StringTok{ }\KeywordTok{t}\NormalTok{(temprotate }\OperatorTok\StringTok{ }\NormalTok{transformation_mat)}
\NormalTok{  \}}\ControlFlowTok{else}\NormalTok{ \{}
    \KeywordTok{print}\NormalTok{ (}\StringTok{"Invalid rotation passed. Returning original, unchanged matrix"}\NormalTok{)}
\NormalTok{    matrixrotate <-}\StringTok{ }\NormalTok{myMatrix}
\NormalTok{  \}}
  
  \KeywordTok{return}\NormalTok{(matrixrotate)}
\NormalTok{\}}
\end{Highlighting}
\end{Shaded}

\hypertarget{origina-matrix}{%
\subsection{Origina Matrix}\label{origina-matrix}}

\begin{Shaded}
\begin{Highlighting}[]
\NormalTok{original_matrix <-}\StringTok{  }\KeywordTok{matrix}\NormalTok{ (}\KeywordTok{c}\NormalTok{ ( }\DecValTok{1}\NormalTok{, }\DecValTok{0}\NormalTok{, }\DecValTok{2}\NormalTok{,}
                               \DecValTok{0}\NormalTok{, }\DecValTok{3}\NormalTok{, }\DecValTok{0}\NormalTok{,}
                               \DecValTok{4}\NormalTok{, }\DecValTok{0}\NormalTok{, }\DecValTok{5}\NormalTok{), }\DataTypeTok{nrow=}\DecValTok{3}\NormalTok{, }\DataTypeTok{byrow=}\NormalTok{T);}



\NormalTok{original_matrix }
\end{Highlighting}
\end{Shaded}

\begin{verbatim}
##      [,1] [,2] [,3]
## [1,]    1    0    2
## [2,]    0    3    0
## [3,]    4    0    5
\end{verbatim}

\hypertarget{rotate-matrix-90-degrees}{%
\subsection{\texorpdfstring{\textbf{Rotate Matrix 90
Degrees}}{Rotate Matrix 90 Degrees}}\label{rotate-matrix-90-degrees}}

\begin{Shaded}
\begin{Highlighting}[]
\NormalTok{rotateMatrix90 <-}\StringTok{ }\KeywordTok{transposeMatrix}\NormalTok{(original_matrix,}\DecValTok{90}\NormalTok{)}
\NormalTok{rotateMatrix90}
\end{Highlighting}
\end{Shaded}

\begin{verbatim}
##      [,1] [,2] [,3]
## [1,]    2    0    5
## [2,]    0    3    0
## [3,]    1    0    4
\end{verbatim}

\hypertarget{rotate-matrix-180-degrees}{%
\subsection{\texorpdfstring{\textbf{Rotate Matrix 180
Degrees}}{Rotate Matrix 180 Degrees}}\label{rotate-matrix-180-degrees}}

\begin{Shaded}
\begin{Highlighting}[]
\NormalTok{rotateMatrix180 <-}\StringTok{ }\KeywordTok{transposeMatrix}\NormalTok{(original_matrix,}\DecValTok{180}\NormalTok{)}
\NormalTok{rotateMatrix180}
\end{Highlighting}
\end{Shaded}

\begin{verbatim}
##      [,1] [,2] [,3]
## [1,]    5    0    4
## [2,]    0    3    0
## [3,]    2    0    1
\end{verbatim}

\hypertarget{rotate-matrix-270-degrees}{%
\subsection{\texorpdfstring{\textbf{Rotate Matrix 270
Degrees}}{Rotate Matrix 270 Degrees}}\label{rotate-matrix-270-degrees}}

\begin{Shaded}
\begin{Highlighting}[]
\NormalTok{rotateMatrix270 <-}\StringTok{ }\KeywordTok{transposeMatrix}\NormalTok{(original_matrix,}\DecValTok{270}\NormalTok{)}
\NormalTok{rotateMatrix270}
\end{Highlighting}
\end{Shaded}

\begin{verbatim}
##      [,1] [,2] [,3]
## [1,]    4    0    1
## [2,]    0    3    0
## [3,]    5    0    2
\end{verbatim}

\newpage

\hypertarget{question-2}{%
\section{\texorpdfstring{\textbf{Question
2}}{Question 2}}\label{question-2}}

Recreate the graphic for the IRIS Data Set using R. Same titles, same
scales, same colors.

Answer: I imported the built-in IRIS data set and grouped IRIS species
and plotted the correlation between the variables

\begin{Shaded}
\begin{Highlighting}[]
\KeywordTok{class}\NormalTok{(iris)}
\end{Highlighting}
\end{Shaded}

\begin{verbatim}
## [1] "data.frame"
\end{verbatim}

\includegraphics{week_03_homework_files/figure-latex/iris plot-1.pdf}

\newpage

\hypertarget{question-3}{%
\section{\texorpdfstring{\textbf{Question
3}}{Question 3}}\label{question-3}}

Right 2-3 sentences concisely defining the IRIS Data Set

Answer: The IRIS data set, sometimes called the Fishers or Anderson data
set. Fisher was a British statistician and biologist and published the
results of the IRIS flower report in 1936. The IRIS dataset measured the
flowers sepal length, sepal width, petal length, and pedal length over a
set of 50 flowers and 3 types of species (setosa, versicolor, and
virginica). The data set is the rock child for begginers learning data
science, hence it is built into R's core language engine.

\newpage

\hypertarget{question-4}{%
\section{Question 4}\label{question-4}}

Import ``personality-raw.txt'' into R. Remove the V00 column. Create two
new columns from the current column ``date\_test''

Below is the R code I developed to import the personality data and
wrangle the data's date\_test column into 2 additional columns, year and
week. These columns were sorted and used to find unique md5 email
strings. I then extracted the md5 email string for
``\href{mailto:monte.shaffer@gmail.com}{\nolinkurl{monte.shaffer@gmail.com}}''
and did a summary of the data set using custom functions seen below.

\begin{Shaded}
\begin{Highlighting}[]
\CommentTok{#source("MyFunctions.R")}

\NormalTok{personality_df <-}\StringTok{ }\KeywordTok{read.table}\NormalTok{(}\StringTok{"personality-raw.txt"}\NormalTok{, }\DataTypeTok{header=}\OtherTok{TRUE}\NormalTok{, }\DataTypeTok{sep =}  \StringTok{"|"}\NormalTok{)}

\CommentTok{#copy data frame}
\NormalTok{modified_df <-}\StringTok{ }\NormalTok{personality_df}

\CommentTok{#remove column $V00}
\NormalTok{modified_df}\OperatorTok{$}\NormalTok{V00 <-}\StringTok{ }\OtherTok{NULL} 

\CommentTok{#split the data_test column into year and week}
\NormalTok{year_time_split <-}\StringTok{ }\KeywordTok{str_split_fixed}\NormalTok{(modified_df}\OperatorTok{$}\NormalTok{date_test, }\StringTok{"}\CharTok{\textbackslash{}\textbackslash{}}\StringTok{s+"}\NormalTok{, }\DecValTok{2}\NormalTok{)}
\NormalTok{year_string <-}\StringTok{  }\KeywordTok{as.Date}\NormalTok{(year_time_split[,}\DecValTok{1}\NormalTok{],}\StringTok{'%m/%d/%Y'}\NormalTok{)}
\NormalTok{year_numeric <-}\StringTok{ }\KeywordTok{as.numeric}\NormalTok{(}\KeywordTok{format}\NormalTok{(year_string,}\StringTok{'%Y'}\NormalTok{))}

\NormalTok{modified_df[}\StringTok{"year"}\NormalTok{] <-}\StringTok{ }\NormalTok{year_numeric}
\NormalTok{modified_df[}\StringTok{"week"}\NormalTok{] <-}\StringTok{ }\KeywordTok{as.numeric}\NormalTok{(}\KeywordTok{strftime}\NormalTok{((year_string), }\DataTypeTok{format =} \StringTok{"%V"}\NormalTok{))}


\CommentTok{# sort data frame by year then week}
\NormalTok{sorted_df <-}\StringTok{ }\KeywordTok{arrange}\NormalTok{(modified_df,}\KeywordTok{desc}\NormalTok{(modified_df}\OperatorTok{$}\NormalTok{year), }\KeywordTok{desc}\NormalTok{(modified_df}\OperatorTok{$}\NormalTok{week))}

\CommentTok{#remove unique email address}
\NormalTok{unique_vec <-}\StringTok{ }\KeywordTok{unique}\NormalTok{(modified_df}\OperatorTok{$}\NormalTok{md5_email)}
\KeywordTok{length}\NormalTok{(unique_vec)}
\end{Highlighting}
\end{Shaded}

\begin{verbatim}
## [1] 678
\end{verbatim}

\begin{Shaded}
\begin{Highlighting}[]
\NormalTok{unique_df <-}\StringTok{ }\NormalTok{sorted_df }\OperatorTok\StringTok{ }\KeywordTok{distinct}\NormalTok{(md5_email, }\DataTypeTok{.keep_all =} \OtherTok{TRUE}\NormalTok{)}

\CommentTok{#write unique email dataframe to text file, pipe delimited}
\KeywordTok{write.table}\NormalTok{(unique_df, }\DataTypeTok{file =} \StringTok{"personality-clean.txt"}\NormalTok{, }\DataTypeTok{sep =} \StringTok{"|"}\NormalTok{)}
\end{Highlighting}
\end{Shaded}

\begin{Shaded}
\begin{Highlighting}[]
\CommentTok{#filter for Monte Shaffer email}
\NormalTok{monte_df <-}\StringTok{ }\KeywordTok{filter}\NormalTok{(unique_df, unique_df}\OperatorTok{$}\NormalTok{md5_email }\OperatorTok{==}\StringTok{ "b62c73cdaf59e0a13de495b84030734e"}\NormalTok{)}

\CommentTok{#remove all columns not starting with "V"}
\NormalTok{monte_df_clean <-}\StringTok{ }\NormalTok{monte_df }\OperatorTok\StringTok{ }\NormalTok{dplyr}\OperatorTok{::}\StringTok{ }\KeywordTok{select}\NormalTok{(}\KeywordTok{starts_with}\NormalTok{(}\StringTok{"V"}\NormalTok{))}

\CommentTok{#Transpose the vector from row to column format}
\NormalTok{monte_df_clean_t <-}\StringTok{ }\KeywordTok{t}\NormalTok{(monte_df_clean)}

\CommentTok{#Summarize the data set}
\KeywordTok{doSummary}\NormalTok{(monte_df_clean_t)}
\end{Highlighting}
\end{Shaded}

\begin{verbatim}
## Record lengeth= 60 
## Number of NA's= 0 
## Mean= 3.48 
## Median= 3.4 
## Mode results= 4.2 
## Naive sum =  208.8   
## Naive sum squard =   771.04  
## Naive variance = 0.7528136   
## 2-pass sum = 208.8   
## 2-pass sum squard =  771.04  
## 2-pass variance =    0.7528136   
## Built-in Standard deviation function= 0.8676483
\end{verbatim}

\begin{verbatim}
## Warning in plot.window(...): "pxlab" is not a graphical parameter
\end{verbatim}

\begin{verbatim}
## Warning in plot.xy(xy, type, ...): "pxlab" is not a graphical parameter
\end{verbatim}

\begin{verbatim}
## Warning in axis(side = side, at = at, labels = labels, ...): "pxlab" is not a
## graphical parameter

## Warning in axis(side = side, at = at, labels = labels, ...): "pxlab" is not a
## graphical parameter
\end{verbatim}

\begin{verbatim}
## Warning in box(...): "pxlab" is not a graphical parameter
\end{verbatim}

\begin{verbatim}
## Warning in title(...): "pxlab" is not a graphical parameter
\end{verbatim}

\includegraphics{week_03_homework_files/figure-latex/myDosummary-1.pdf}

\newpage

\hypertarget{question-5}{%
\section{Question 5}\label{question-5}}

Write functions for doSummary and sampleVariance and doMode. test these
functions in your homework on the
``\href{mailto:monte.shaffer@gmail.com}{\nolinkurl{monte.shaffer@gmail.com}}''
record from the clean dataset. Report your findings. For this
``\href{mailto:monte.shaffer@gmail.com}{\nolinkurl{monte.shaffer@gmail.com}}''
record, also create z-scores. Plot(x,y) where x is the raw scores for
``\href{mailto:monte.shaffer@gmail.com}{\nolinkurl{monte.shaffer@gmail.com}}''
and y is the z-scores from those raw scores. Include the plot in your
assignment, and write 2 sentences describing what pattern you are seeing
and why this pattern is present.

I created R functions, ``doSummary'', ``sampleVariance'' and ``doMode''

Dataset findings: * original dataset had 838 records * clean dataset had
678 records * 2-pass and Naive algorithms produced the same results for
variance

Zscores plot: * The Z scores plot. A Z score, also called the ``Standard
Score'', measures how many standard deviations below or above the
population mean a raw score is. A Z score of 0 (``zero'') is exactly
average. A Z score of 1 is 1 stadard deviation above the mean. A Z score
of -2 is -2 standard deviations below the mean.

The average for
``\href{mailto:monte.shaffer@gmail.com}{\nolinkurl{monte.shaffer@gmail.com}}''
was 3.4 and the Z score plot shows a Z score of 0 at 3.4.

\newpage

\hypertarget{question-6}{%
\section{Question 6}\label{question-6}}

Compare Will Smith and Denzel Washington. You will have to create a new
variable \$millions.2000 that converts each movie's \$millions based on
the \$year of the movie, so all dollars are in the same time frame. You
will need inflation data from about 1980-2020 to make this work.

I created a custom function seen below to convert the actors revenue and
adjust for inflation. The inflation year is passed in by the user so its
flexible by any given year from 1920-2020

\begin{Shaded}
\begin{Highlighting}[]
  \CommentTok{#source("MyFunctions.R")}

\NormalTok{  nmid =}\StringTok{ "nm0000226"}\NormalTok{;}
\NormalTok{  will =}\StringTok{ }\KeywordTok{grabFilmsForPerson}\NormalTok{(nmid);}
\NormalTok{  a <-}\StringTok{ }\NormalTok{will}\OperatorTok{$}\NormalTok{movies}\FloatTok{.50}
  \CommentTok{#plot(will$movies.50[,c(10)]);}
  \CommentTok{#boxplot(will$movies.50$millions); }
\NormalTok{  widx =}\StringTok{  }\KeywordTok{which.max}\NormalTok{(will}\OperatorTok{$}\NormalTok{movies}\FloatTok{.50}\OperatorTok{$}\NormalTok{millions);}
\NormalTok{  will}\OperatorTok{$}\NormalTok{movies}\FloatTok{.50}\NormalTok{[widx,];}
\end{Highlighting}
\end{Shaded}

\begin{verbatim}
##    rank   title      ttid year rated minutes                      genre ratings
## 15   15 Aladdin tt6139732 2019    PG     128 Adventure, Family, Fantasy       7
##    metacritic  votes millions
## 15         53 216689   355.56
\end{verbatim}

\begin{Shaded}
\begin{Highlighting}[]
  \CommentTok{#summary(will$movies.50$year);  # bad boys for life ... did data change?}
 
  \CommentTok{#Denzel Washington}
\NormalTok{  nmid =}\StringTok{ "nm0000243"}\NormalTok{;}
\NormalTok{    denzel =}\StringTok{ }\KeywordTok{grabFilmsForPerson}\NormalTok{(nmid);}
    \CommentTok{#plot(denzel$movies.50[,c(10)]);}
  \CommentTok{#boxplot(denzel$movies.50$millions);}
\NormalTok{    didx =}\StringTok{  }\KeywordTok{which.max}\NormalTok{(denzel}\OperatorTok{$}\NormalTok{movies}\FloatTok{.50}\OperatorTok{$}\NormalTok{millions);}
\NormalTok{    denzel}\OperatorTok{$}\NormalTok{movies}\FloatTok{.50}\NormalTok{[didx,];}
\end{Highlighting}
\end{Shaded}

\begin{verbatim}
##   rank             title      ttid year rated minutes                   genre
## 1    1 American Gangster tt0765429 2007     R     157 Biography, Crime, Drama
##   ratings metacritic  votes millions
## 1     7.8         76 384191   130.16
\end{verbatim}

\begin{Shaded}
\begin{Highlighting}[]
    \CommentTok{#summary(denzel$movies.50$year);}
    
    \KeywordTok{par}\NormalTok{(}\DataTypeTok{mfrow=}\KeywordTok{c}\NormalTok{(}\DecValTok{1}\NormalTok{,}\DecValTok{2}\NormalTok{))}
    \KeywordTok{boxplot}\NormalTok{(will}\OperatorTok{$}\NormalTok{movies}\FloatTok{.50}\OperatorTok{$}\NormalTok{millions, }\DataTypeTok{main=}\NormalTok{will}\OperatorTok{$}\NormalTok{name, }\DataTypeTok{ylim=}\KeywordTok{c}\NormalTok{(}\DecValTok{0}\NormalTok{,}\DecValTok{360}\NormalTok{), }\DataTypeTok{ylab=}\StringTok{"Raw Millions"}\NormalTok{ )}
    \KeywordTok{boxplot}\NormalTok{(denzel}\OperatorTok{$}\NormalTok{movies}\FloatTok{.50}\OperatorTok{$}\NormalTok{millions, }\DataTypeTok{main=}\NormalTok{denzel}\OperatorTok{$}\NormalTok{name, }\DataTypeTok{ylim=}\KeywordTok{c}\NormalTok{(}\DecValTok{0}\NormalTok{,}\DecValTok{360}\NormalTok{), }\DataTypeTok{ylab=}\StringTok{"Raw Millions"}\NormalTok{ )}
\end{Highlighting}
\end{Shaded}

\includegraphics{week_03_homework_files/figure-latex/IMBD analysis-1.pdf}

\begin{Shaded}
\begin{Highlighting}[]
    \CommentTok{#read inflation data}
\NormalTok{    inflation_df <-}\StringTok{ }\KeywordTok{read.table}\NormalTok{(}\StringTok{"inflation.txt"}\NormalTok{, }\DataTypeTok{header=}\OtherTok{TRUE}\NormalTok{, }\DataTypeTok{sep =}  \StringTok{"|"}\NormalTok{)}
    
    \CommentTok{#calculate inflation adjustment for Will Smith movie revenue}
\NormalTok{    will}\OperatorTok{$}\NormalTok{millions}\FloatTok{.2000}\NormalTok{ <-}\StringTok{ }\KeywordTok{convert_inflation}\NormalTok{(will}\OperatorTok{$}\NormalTok{movies}\FloatTok{.50}\NormalTok{,}\DecValTok{2000}\NormalTok{, inflation_df)}
    \CommentTok{#will$millions.2000}
    
    \CommentTok{#calculate inflation adjustment for Denzel Washington movie revenue}
\NormalTok{    denzel}\OperatorTok{$}\NormalTok{millions}\FloatTok{.2000}\NormalTok{ <-}\StringTok{ }\KeywordTok{convert_inflation}\NormalTok{(denzel}\OperatorTok{$}\NormalTok{movies}\FloatTok{.50}\NormalTok{,}\DecValTok{2000}\NormalTok{, inflation_df)}
    \CommentTok{#denzel$millions.2000}
    
    \CommentTok{#Plot new box plots with inflation adjusted revenues}
    \KeywordTok{par}\NormalTok{(}\DataTypeTok{mfrow=}\KeywordTok{c}\NormalTok{(}\DecValTok{1}\NormalTok{,}\DecValTok{2}\NormalTok{))}
    \KeywordTok{boxplot}\NormalTok{(will}\OperatorTok{$}\NormalTok{millions}\FloatTok{.2000}\NormalTok{, }\DataTypeTok{main=}\NormalTok{will}\OperatorTok{$}\NormalTok{name, }\DataTypeTok{ylim=}\KeywordTok{c}\NormalTok{(}\DecValTok{0}\NormalTok{,}\DecValTok{600}\NormalTok{), }\DataTypeTok{ylab=}\StringTok{"Inflation adjusted Millions"}\NormalTok{ )}
    \KeywordTok{boxplot}\NormalTok{(denzel}\OperatorTok{$}\NormalTok{millions}\FloatTok{.2000}\NormalTok{, }\DataTypeTok{main=}\NormalTok{denzel}\OperatorTok{$}\NormalTok{name, }\DataTypeTok{ylim=}\KeywordTok{c}\NormalTok{(}\DecValTok{0}\NormalTok{,}\DecValTok{600}\NormalTok{), }\DataTypeTok{ylab=}\StringTok{"Inflation adjusted Millions"}\NormalTok{ )}
\end{Highlighting}
\end{Shaded}

\includegraphics{week_03_homework_files/figure-latex/IMBD analysis-2.pdf}

\newpage

\hypertarget{question-7}{%
\section{Question 7}\label{question-7}}

Build side-by-side box plots on several of the variables (including \#6)
to compare the two movie stars. After each box plot, write 2+ sentence
describing what you are seeing, and what conclusions you can logically
make. You will need to review what the box plot is showing with the box
portion, the divider in the box, and the whiskers.

I created the following box plots, Will Smith vs.~Denzel Washington: *
Box plot for ratings The resuls of the ratings showed Denzel had a
better rating over his career than Will S. Will S. does have one outlier
rating of 2.3(Student of the Year 2). Denzel has a tight box plot, which
showes his movies ranking have been consistent over his career

\begin{itemize}
\tightlist
\item
  Box plot for movie length(minutes) The results of the movie minutes
  showed Denzel tends to make longer movies than Will S. Will S. box
  plot shows a taller whiskers which indicates he has several movies he
  made that were outside his normal movie length. These movies where
  ``Ali'' and ``Bad Boys II''
\end{itemize}

\begin{Shaded}
\begin{Highlighting}[]
  \CommentTok{#Plot movie rankings}
    \KeywordTok{par}\NormalTok{(}\DataTypeTok{mfrow=}\KeywordTok{c}\NormalTok{(}\DecValTok{1}\NormalTok{,}\DecValTok{2}\NormalTok{))}
    \KeywordTok{boxplot}\NormalTok{(will}\OperatorTok{$}\NormalTok{movies}\FloatTok{.50}\OperatorTok{$}\NormalTok{ratings, }\DataTypeTok{main=}\NormalTok{will}\OperatorTok{$}\NormalTok{name, }\DataTypeTok{ylim=}\KeywordTok{c}\NormalTok{(}\DecValTok{0}\NormalTok{,}\DecValTok{10}\NormalTok{), }\DataTypeTok{ylab=}\StringTok{"Ratings"}\NormalTok{ )}
    \KeywordTok{boxplot}\NormalTok{(denzel}\OperatorTok{$}\NormalTok{movies}\FloatTok{.50}\OperatorTok{$}\NormalTok{ratings, }\DataTypeTok{main=}\NormalTok{denzel}\OperatorTok{$}\NormalTok{name, }\DataTypeTok{ylim=}\KeywordTok{c}\NormalTok{(}\DecValTok{0}\NormalTok{,}\DecValTok{10}\NormalTok{), }\DataTypeTok{ylab=}\StringTok{"Ratings"}\NormalTok{ )}
\end{Highlighting}
\end{Shaded}

\includegraphics{week_03_homework_files/figure-latex/boxplot summary of IMBD-1.pdf}

\begin{Shaded}
\begin{Highlighting}[]
    \CommentTok{#plot movie minutes}
    \KeywordTok{par}\NormalTok{(}\DataTypeTok{mfrow=}\KeywordTok{c}\NormalTok{(}\DecValTok{1}\NormalTok{,}\DecValTok{2}\NormalTok{))}
    \KeywordTok{boxplot}\NormalTok{(will}\OperatorTok{$}\NormalTok{movies}\FloatTok{.50}\OperatorTok{$}\NormalTok{minutes, }\DataTypeTok{main=}\NormalTok{will}\OperatorTok{$}\NormalTok{name, }\DataTypeTok{ylim=}\KeywordTok{c}\NormalTok{(}\DecValTok{60}\NormalTok{,}\DecValTok{180}\NormalTok{), }\DataTypeTok{ylab=}\StringTok{"Minutes"}\NormalTok{ )}
    \KeywordTok{boxplot}\NormalTok{(denzel}\OperatorTok{$}\NormalTok{movies}\FloatTok{.50}\OperatorTok{$}\NormalTok{minutes, }\DataTypeTok{main=}\NormalTok{denzel}\OperatorTok{$}\NormalTok{name, }\DataTypeTok{ylim=}\KeywordTok{c}\NormalTok{(}\DecValTok{60}\NormalTok{,}\DecValTok{180}\NormalTok{), }\DataTypeTok{ylab=}\StringTok{"Minutes"}\NormalTok{ )}
\end{Highlighting}
\end{Shaded}

\includegraphics{week_03_homework_files/figure-latex/boxplot summary of IMBD-2.pdf}




\newpage
\bibliographystyle{./../biblio/chicago}
\bibliography{./../biblio/master} % this is master.bib located in the biblio subfolder ... you could have multiple files here if need be ... one is easier ...


\end{document}